\documentclass[12pt]{article}
\title{Suffering Through Sanctions: An Application of Utilitarian Ethics to Coercive Economic Punishments}
\author{Word Count: 100}
\date{\today}

\usepackage{setspace}
\usepackage[utf8]{inputenc}
\usepackage{CormorantGaramond}
\usepackage[hidelinks]{hyperref}
\usepackage{indentfirst}
\usepackage{fancyhdr}
\usepackage{lastpage}
\usepackage{microtype}
\usepackage{blindtext}
\usepackage[margin=1in]{geometry}
\usepackage[american]{babel}
\usepackage{csquotes}
\usepackage[style=authoryear-comp]{biblatex}
\addbibresource{library.bib}
\pagestyle{fancy}
\fancyhf{}
\rhead{Suffering Through Sanctions \thepage}
\setlength{\headheight}{15pt}
\linespread{2}
\raggedright{}
\setlength\parindent{0.5in}

\begin{document}
\pagenumbering{gobble}
\maketitle
\newpage
\pagenumbering{arabic}

\section*{Introduction}
In the modern and globalized world we live in where every action of the various leaders and governments across the globe can be instantly seen and critiqued by nearly any individual person at a moments notice, the glaring differences between nations are becoming increasingly obvious.
In fact, due to factors such as censorship and unequal access to the internet, there are even cases where any average American has the ability to learn more about the proceedings of a foreign government than even the citizens living in that country.
This shift in information has led to a corresponding shift in the questions we face, where we don't look whether or not a human rights abuse or other form of suffering is occurring in a remote part of the world, but rather what the international community ought to do about it.
In this paper, I will argue that we all have a collective moral obligation to reduce the total amount of suffering in the world, and that this obligation ought to take precedent over our desire to create additional happiness or pleasure.
More broadly, this means that we have a moral responsibility to avoid causing additional suffering in the world, even when we initially believe that it will lead to a noble goal.
In order to apply this theory to public policy, I will argue that coercive economic sanctions as a means of enforcing international values and norms are unjust, and that they instead need to be replaced with diplomatic negotiations and positive economic alternatives.

\section*{Moral Obligations}
First, before looking at any public policy issues, I will try to answer the more general question of what moral obligations we have other human beings.
I believe that the best approach to answering this question can be found in the realm of utilitarianism, which broadly says that we ought to promote happiness and reduce suffering.
The first aspect of my theory can be found in Peter Singer's cosmopolitan interpretation of utilitarianism, which is both universal, as it the number of people who could perform the positive-utility act is irrelevant, and global, as the suffering of one individual cannot be proffered over the suffering of another regardless of proximity or distance \autocite[154]{widdows2012}.
This conclusion is derived from the shallow pond example, which asks us to consider if we would be morally obliged to save a drowning child from a pond, even if it meant getting your clothes wet and muddy.
Since most people would presumably agree that the life of the child outweighs the cost of the clothes, Singer colludes that ``if it is in our power to prevent something bad from happening, without thereby sacrificing anything of comparable moral importance, we ought, morally, to do it \autocite[153]{widdows2012}.'' 
Singer primarily focuses this account of justice on humanitarian aid, where he states that well-off individuals have a moral responsibility to donate to needy people outside of the geographic community \autocite[155]{widdows2012}.
He also extends this individual commitment to the level of the nation-state, claiming that rich nations ought to follow their ethical responsibilities to the global community \autocite[15]{singer2016}.

The second part of my theory goes against the classical utilitarian idea that pain and pleasure are direct opposites, and that pain is simply a negative version of pleasure. Instead, I believe that our obligation to reduce suffering is stronger than our responsibility to promote happiness.
This view is reflected by Karl Popper in the book \textit{The Open Society and Its Enemies}, where he states that one person's pain cannot always be outweighed by their own pleasure, and by extension cannot by outweighed by the pleasure of another person. \autocite[285]{popper2002}.
This view, which more generally advocates for the ``least amount of avoidable suffering for all'', has been given the term \textit{negative utilitarianism} \autocite[542]{smart1958}.
Although there are many varieties of negative utilitarianism that can be discussed at length, I will be looking at it more generally as a moral framework and approach to utilitarian public policy as opposed to a complete theory of justice.
Therefore, I believe that every moral agent has an obligation to reduce the suffering of other sentient creatures as long as it doesn't require the creation of a comparable amount of suffering.
To additionally clarify, although I believe that the reduction of the suffering of animals is also morally required by negative utilitarianism under this, I will be primarily speaking about the suffering of humans as they are the only ones that are generally considered to be relevant in the realm of public policy.

\newpage
\printbibliography{}
\end{document}
