\documentclass[12pt]{article}
\title{Suffering Through Sanctions: An Application of Negative Utilitarianism to Coercive Economic Punishments}
\author{Word Count: 3661}
\date{\today}

\usepackage{setspace}
\usepackage[utf8]{inputenc}
\usepackage{CormorantGaramond}
\usepackage[hidelinks]{hyperref}
\usepackage{indentfirst}
\usepackage{fancyhdr}
\usepackage{lastpage}
\usepackage{microtype}
\usepackage{blindtext}
\usepackage[margin=1in]{geometry}
\usepackage[american]{babel}
\usepackage{csquotes}
\usepackage[style=mla-new]{biblatex}
\addbibresource{library.bib}
\pagestyle{fancy}
\fancyhf{}
\rhead{Suffering Through Sanctions \thepage}
\setlength{\headheight}{15pt}
\linespread{2}
\raggedright{}
\setlength\parindent{0.5in}

\begin{document}
\pagenumbering{gobble}
\maketitle
\newpage
\pagenumbering{arabic}

\section*{Introduction}
In the modern and globalized world we live in where every action of the various leaders and governments across the globe can be instantly seen and critiqued by nearly any individual person at a moments notice, the glaring differences between nations are becoming increasingly obvious.
In fact, due to factors such as censorship and unequal access to the internet, there are even cases where any average American has the ability to learn more about the proceedings of a foreign government than even the citizens living in that country.
This shift in information has led to a corresponding shift in the questions we face, where we don't look whether or not a human rights abuse or other form of suffering is occurring in a remote part of the world, but rather what the international community ought to do about it.
In this paper, I will argue that we all have a collective moral obligation to reduce the total amount of suffering in the world, and that this obligation ought to take precedent over our desire to create additional happiness or pleasure.
More broadly, this means that we have a moral responsibility to avoid causing additional suffering in the world, even when we initially believe that it will lead to a noble goal.
In order to apply this theory to public policy, I will argue that coercive economic sanctions as a means of enforcing international values and norms are unjust, and that they instead need to be replaced with diplomatic negotiations and positive economic alternatives.

\section*{Moral Obligations}
\subsection*{Negative Utilitarianism}
First, before looking at any public policy issues, I will try to answer the more general question of what moral obligations we have other human beings.
I believe that the best approach to answering this question can be found in the realm of utilitarianism, which broadly says that we ought to promote happiness and reduce suffering.
The first aspect of my theory can be found in Peter Singer's cosmopolitan interpretation of utilitarianism, which is both universal, as it the number of people who could perform the positive-utility act is irrelevant, and global, as the suffering of one individual cannot be proffered over the suffering of another regardless of proximity or distance \autocite[154]{widdows2012}.
This conclusion is derived from the shallow pond example, which asks us to consider if we would be morally obliged to save a drowning child from a pond, even if it meant getting your clothes wet and muddy.
Since most people would presumably agree that the life of the child outweighs the cost of the clothes, Singer colludes that ``if it is in our power to prevent something bad from happening, without thereby sacrificing anything of comparable moral importance, we ought, morally, to do it'' \autocite[153]{widdows2012}.
Singer primarily focuses this account of justice on humanitarian aid, where he states that well-off individuals have a moral responsibility to donate to needy people outside of the geographic community \autocite[155]{widdows2012}.
He also extends this individual commitment to the level of the nation-state, claiming that rich nations ought to follow their ethical responsibilities to the global community \autocite[15]{singer2016}.
I strongly agree with both of these assertions, as I agree that the prevention of suffering ought to be practiced on both the individual and collective levels.
For example, more localized issues such as homelessness and housing might be better solved on an individual level, whereas solving broader systemic issues such as corruption could be more efficient on the level of nations.

The second part of my theory goes against the common utilitarian argument that pain can always be outweighed by a higher level of pleasure.
Instead, I believe that our obligation to reduce suffering is stronger than our responsibility to promote happiness.
This view is reflected by Karl Popper in the book \textit{The Open Society and Its Enemies}, where he states that one person's pain cannot always be outweighed by their own pleasure, and by extension cannot by outweighed by the pleasure of another person. \autocite[285]{popper2002}.
This view, which more generally advocates for the ``least amount of avoidable suffering for all'' has been given the term \textit{negative utilitarianism} \autocite[542]{smart1958}.
I agree with this view due to my personal belief that creating any amount of suffering in order to create an equal amount of pleasure intuitively seems to be undesirable.
Whereas the process of hedonistic calculus would see such an as action as morally neutral, I would see it as morally unjust.
For example, if I were to see someone steal a person's wallet and make them suffer in order to give it to someone else that would receive an equal amount of pleasure as a result, I would still condemn that action.
Additionally, I would even go a step further by saying that some amounts of suffering cannot be outweighed.
If I had the opportunity to torture one person with the most painful experience humanly possible, I would not allow for that even if it meant providing ten people the most pleasurable experience humanly possible.
Although there are many varieties of negative utilitarianism that can be discussed at length, I will be looking at it more generally as a moral framework and approach to utilitarian public policy as opposed to a complete theory of justice.
Therefore, I believe that every moral agent has an obligation to reduce the suffering of other sentient creatures as long as it doesn't require the creation of a comparable amount of suffering.
To additionally clarify, although I believe that the reduction of the suffering of animals is also morally required by negative utilitarianism under this, I will be primarily speaking about the suffering of humans as they are the only ones that are generally considered to be relevant in the realm of public policy.

\subsection*{Nussbaum's Capabilities Approach}
Although I have only spoken about our moral obligations through a utilitarian framework, there are many other views within the broader group of cosmopolitan ethics that have become relevant during modern discussions of public policy and international relations.
One of those views is the capabilities approach, which is primarily defended by the American philosopher Martha Nussbaum.
This theory is based first off of the idea that common measurements of the wellbeing of nations are inadequate when it comes to measuring the lives of all people rather than just a small group of people, and that the measurements also lose sight of what we ought to truly see as valuable in an individual person's life.
For example, prevalent economic measurements such as wealth or GDP per capita will measure the gains in wealth of rich and poor people equally, even though an equal gain in wealth will affect both of their lives in completely different ways.
Individual people also have varying abilities to make use of their material conditions, leading to further inaccuracies \autocite[238]{nussbaum2019}.
Additionally, Nussbaum believes that the more philosophical view that deals with the satisfaction of preferences are also inadequate for a few main reasons.
First, it faces a similar problem to GDP in that it accounts for every preference equally with no regard for distribution.
Next, people in worse circumstances may exhibit adaptive preferences, where they change their personal preferences in accordance to what they believe they can achieve.
This can make it hard to determine what situations people would actually prefer to be in if given the choice, as they might not know what is best for themselves.
Finally, there is the problem of Robert Nozick's experience machine, which makes the point that we would all rather experience a real life rather than having all of our preferences satisfied by a machine, which would suggest that we value more than just our pleasure or preferences \autocite[239]{nussbaum2019}.

As a result of these problems, Nussbaum suggests that we instead measure the welfare of a nation through the ability of the people to access a set of human capabilities.
Nussbaum believes that instead of simply looking at the material conditions of people or their mental states, we ought to examine the parts of life that we deem to be most important or critical in a complete and fulfilling life.
More specifically, we should examine the access people have to \emph{combined capabilities}, which are created through a combination of care, development, and adequate material conditions \autocite[240]{nussbaum2019}.
Nussbaum goes on to propose a preliminary list of ten capabilities, which are known as \emph{The Central Human Capabilities} that need to be available to the people of a nation in order for the nation to have a baseline claim to justice.
This list includes, but  is not limited to, physical factors such as the ability to live a life without dying prematurely and being free from violent assault, abstract capabilities such as the ability to think and reason, and emotional and social factors such as the ability to enjoy recreational activities and associate with others \autocite[242]{nussbaum2019}.
Nussbaum believes that these capabilities can solve the problem of adaptive preferences while still leaving room for social and cultural differences \autocite[243]{nussbaum2019}.
This view differs from the utilitarian view in a few key ways.
First, it looks to evaluate one's quality of life in a more holistic and qualitative way as opposed to the purely quantitative hedonistic calculus of utilitarianism.
As a result of this, it may end up valuing some forms of suffering more than others depending on how the central capabilities are determined.
While this would depend on which capabilities are chosen, an example would be how someone could see physical pain through injury or illness as being somehow worse than a corresponding emotional or mental pain.
Next, while the utilitarian view sees suffering as universally bad, the capabilities approach only requires people to be given the ability to avoid a certain amount of suffering in order to qualify as having a baseline level of justice.
While the utilitarian would aim to reduce suffering as much as possible, the capabilities approach would be conceptually fine with suffering as long as it didn't infringe on access to the central capabilities.

Although Nussbaum does present some valid critiques to the measurements of GDP per capita and satisfaction of preferences, I still believe that the approach of negative utilitarianism is the superior way to promote justice around the world.
First, I believe that the capabilities approach, at least in its current form, is only useful as a baseline to understand the cases where there is not justice.
I would agree that if the capabilities proposed by Nussbaum are not available, there is going to be a large amount of suffering within that nation.
However, in my view, the consideration of justice is not one we can look at solely in terms of baseline, where a nation either does or does not fulfill their basic duties of justice.
Rather, the call for a minimization of suffering can cause two nations to fulfill their duties with different degrees of success.
Additionally, if we are to believe that Nussbaum's proposed capabilities will be similar to the true set of central human capabilities, I believe that they deal too heavily with the promotion of happiness rather than the reduction of suffering.
The most glaring example would be the capability regarding play.
The negative utilitarian theory is based on the idea that there is an asymmetry between pain and pleasure.
As Popper says, while pain always makes a direct appeal for help, there is no call to increase the happiness of someone who is already happy \autocite[284]{popper2002}.
Although this asymmetry makes it obvious that the capabilities regarding safety and freedom from violence are much more important than those regarding recreation, Nussbaum's version of the capability approach doesn't seem to provide a set or system of weights to judge which capabilities to focus on first.

Nussbaum would likely respond to these criticisms by saying that the capabilities approach is not meant to be a complete theory of justice.
In regards the problem of varying degrees of justice, she would likely also agree that two nations that fulfill the basic capacities of the people could differ in the amount of justice present.
These levels could be still be evaluated using access to central capabilities as a tool of measurement.
While I agree that this could theoretically be done, I would need to see more clarity about how the central capabilities would be established, agreed upon, and implemented.
Otherwise, there will be inevitable confusion about the presence of justice, or lack thereof, in nearly any given nation.
While this is a problem with nearly any theory of wellbeing, as even utilitarianism would need to find a way to measure suffering, the capabilities approach is generally much more complex.
Whereas utilitarianism only needs a way to define and measure pain, the capabilities approach would theoretically need to choose the central capabilities, define each of the capabilities, find a way to measure each of the capabilities, and then decide how the weigh each of them against each other.
In regards to the second problem, Nussbaum would likely have a similar response.
Using our reason, we could most likely come up with a more concrete way to weigh the central capabilities against each.
For both of these arguments, I would say that while I see solutions to them as possibilities, I have yet to see that those solutions could feasibly be implemented in a way that wouldn't cause further problems due to bias or subjectivity.

\section*{Application to Economic Sanctions}
\subsection*{Ethical Requirements}
As part of the utilitarian commitment to universality, Singer believes that we need to look past our moral intuitions and prioritization of our local communities, and instead look to our ethical reasoning as a source for justice in the modern globalized world \autocite[14]{singer2016}.
One example of this is through the use of coercive economics sanctions, where one or more nations will withhold trade relations with another nation in order to force them into taking or ceasing some kind of action.
In the paper ``A Peaceful, Silent, Deadly Remedy'', Joy Gordon analyzes the moral legitimacy of broad economic sanctions through the lenses of the just war doctrine, Kantian ethics, and utilitarianism.
In the section regarding utilitarianism, they claim that sanctions are ``a device specifically tailored to harm nations with dependent or weak economies \autocite[135]{gordon1999}.''
Gordon goes on to say that sanctions could easily be justified in a theoretical sense through a utilitarian framework, as long as they were meant to cause more pleasure than pain.
However, this condition can only be met after there is strong evidence that the target nation would comply with the demands of the ones imposing the sanctions, otherwise they are just a mechanism to impose suffering on innocent civilians without ethical justification \autocite[137]{gordon1999}.
After observing many of these same faults, Canadian physician and international affairs researcher Neil Arya claims that we ought to treat economic sanctions in a similar way to military intervention, as in many cases they can they can have the same level of lethality.
Arya proposes that we should use the just war principles of \textit{jus ad bellum} (just causes of war) and \textit{jus in bello} (justice during war) as baselines, as well as critical examinations of factors such as prospects of success and viability of other measures, in order to decide whether or not an economic sanction is justifiable \autocite[35]{arya2008}.
Both of these perspectives are open to the idea of sanctions, but would require a high burden of proof and justification before their implementation.

Based on these considerations, we can now apply the theory of negative utilitarianism to economic sanctions.
I believe that in most cases, economic sanctions are a direct violation of the duties of respect that we owe to all creatures as a result of their ability to suffer.
However, in order to avoid missing any chances to reduce further suffering, I believe it is important to still evaluate the potential costs and effects of each proposed economic sanction on an individual level.
In order to accomplish this, I would like to propose three rules.
First, an economic sanction is only morally just if the amount of suffering that is likely to be reduced is larger than the amount of suffering that is likely to be caused by its implementation.
Second, if an economic sanction will cause suffering, the suffering must be reduced and/or targeted in a way that will avoid any unnecessary suffering.
Finally, an economic sanction should only be implemented in the case that every other alternative course of action would result in more total suffering.
A consideration for all of these rules is that Singer's commitment to universality must be maintained in the calculation of suffering.
For instance, it would not be permissible to implement a sanction that causes life-saving medicine to be denied from five people in the target nation, even if that sanction would prevent four American soldiers from dying in combat.
Another result of these rules would be that this calculation would likely line up with the procedures proposed by both Gordon and Arya, as both of them would involve needing strong evidence in order to commit to the action of implementing an economic sanction.

\subsection*{Empirical Backing}
In order to understand what should be done about economic sanctions in the future, it is important to examine the effects of various instances economic sanctions throughout history.
First, previous examples of coercive economic sanctions have resulted in the deterioration of the living standards and health of vulnerable people while having only minor impacts on the powerful.
In a paper from 1999 which describes the humanitarian effects of economic sanctions against Cuba, Yugoslavia, and Iraq, Ulrich Gottstein gives the example of the economic sanctions early 1990s that were placed in response to the Castro Regime’s `extraordinary violations of human rights'. 
In one round of these sanctions, the 1992 Cuban Democracy Act restricted \emph{all} trade between Cuba and the US\@.
It also placed restrictions on trade with European firms and ships that cooperated with or traded with Cuba \autocite[273]{gottstein1999}.
As a result, the food supply had diminished to a point where shelves in shops were completely empty.
Additionally, this forced Cuban people to buy vital medicines, medical equipment, and materials through long production chains in Europe and China that resulted in costs exceeding over ten times the expected price \autocite[274]{gottstein1999}.
The positive effects of the sanctions were virtually non-existent, as the blockade ``[did] not hit the political figures but only innocent people \autocite[275]{gottstein1999}''
Similar effects can be found in the sanctions imposed on Haiti by the United States in 1991.
Shortages of oil led to an increase in wood fuel production, which lead to widespread deforestation.
Shortages of pesticides and fertilizers led to lower crop yields and increased hunger.
It is even estimated that manufacturing 240,000 jobs were lost due to reduce trade and economic cooperation, all while the incomes of military and government leaders were relatively unaffected \autocite[27]{clawson1993}.
This research provides evidence towards the idea that economic sanctions would be unlikely to meet the standard that is required by negative utilitarianism.
Since there is oftentimes more suffering caused as a result of the sanctions than the amount of suffering relieved by their effects, the sanctions in these examples would be considered morally unjust.

Another negative effect of coercive economics sanctions is an erosion of democracy and liberty in the target nations.
A study by Dursun Peksen and A. Cooper Drury found that economic sanctions, in specifications of limited and extended, have a statistically significant negative impact on democracy that is comparable to even civil war in both the short-term and the long-term.
In the long term, economic sanctions as a whole tend to have an even larger negative effect on democracy than civil war \autocite[255, 257]{peksen2010}.
First, since most nations that get targeted by economic sanctions do not have robust economies, the sanctions lead to a reduction in wealth among the people of nation.
This gives an authoritarian regime the opportunity to take control of or severely intervene in the economy in order to rebuild wealth.
An example of this is came from UN Security Council sanctions against Rhodesia, where tobacco farmers turned over control of the tobacco market to the Rhodesian government in order to avoid a price war that was predicted to occur due to worsened economic conditions.
The Rhodesian government used this new power to redirect wealth away from the farmers and towards the industrialists who tended to give a higher amount of support to the Smith regime, therefore turning the poor economic conditions into an opportunity to consolidate political power into the hands of a small group of people \autocite[244]{peksen2010}.
Similarly to the sanctions in Cuba and Haiti, the sanctions implemented against Rhodesia seemed to create a higher amount of suffering.

In order to apply these findings to policy prescriptions, I believe that the United States and other wealthy nations ought to substantially reduce their usage of economic sanctions as a means of enforcing international norms and creating change.
Looking at the aforementioned cases of sanctions against the nations of Cuba, Haiti, and Rhodesia, I believe that none of them would pass the requirements that I previously proposed.
Looking strictly at the amount of suffering, existing research seems to suggest that they all caused more suffering through the harmful economic and democratic effects than the amount of suffering relieved by pressuring the governments to create change.
Although those are by no means the only times that sanctions have been tried, they seem to be reflective of a broader lack of understanding of the moral responsibilities of powerful nations.
Moving forward, I believe that we are required to either reform economic sanctions in a way that will increase their effectiveness and reduce the amount of suffering caused, or look to other diplomatic measures as an alternative.
While Popper is correct in his notion that suffering makes a direct appeal for help, we must consider that the ways in which we relieve that suffering shouldn't be through public policy that will increase further suffering.

\newpage
\printbibliography{}
\end{document}
